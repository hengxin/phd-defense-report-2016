%%%%%%%%%%%%%%%%%%%%%%%%%%%%%%%%%%%%%%%%%%%%%%%%%%%%%%%%%%%%%%%%%%%%%%%%%%%%%%%	
\section{未来工作}

\begin{frame}{工作总结}
  \fignocaption{width = 0.80\textwidth}{figures/work-1+3-framework.pdf}
\end{frame}

\begin{frame}{未来工作}
  \textcolor{blue}{多样化, 可定制; 精细化, 可度量:}
  \begin{enumerate}
    \item 支持``多写''的 atomic 变量 (扩展 2AM 工作)
      \uncover<2,5>
      {
      \begin{itemize}
        \setlength\itemsep{3pt}
        \item \textcolor{blue}{\bf 前提:} 读操作只需一轮网络通信 \emph{(fast read)}
        \item \textcolor{blue}{\bf 理论问题:} 是否存在允许 fast read 的 ($k$-)atomicity 算法?
        \item \question{可度量:} 如何定义\&量化 $p$AM ($p:$ probabilistic)?
      \end{itemize}
      }
    \vspace{0.30cm}
    \item 支持 P2P 架构的``一致性可定制''实现
      \uncover<3,5>
      {
      \begin{itemize}
        \setlength\itemsep{3pt}
        \item \textcolor{blue}{\bf 已有工作:} 非事务, 可定制, master-slave 架构 
          \citeinbeamer{Terry}{SOSP}{13}
        \item \textcolor{blue}{\bf 动机:} Cassandra 采用 P2P 架构 
          \citeinbeamer{Facebook}{SIGOPS OSR}{10}
        \item \question{可定制:} 一致性模型的兼容性与重定义
      \end{itemize}
      }
    \vspace{0.30cm}
    \item 事务与非事务一致性模型的统一框架
      \uncover<4,5>
      {
      \begin{itemize}
        \item \textcolor{blue}{\bf 异:} ``all-or-none'' 语义
        \item \textcolor{blue}{\bf 同:} 操作间序关系
        \item \question{多样化:} 更丰富, 更结构化的一致性模型
      \end{itemize}
      }
  \end{enumerate}
\end{frame}

\begin{frame}[noframenumbering]
  \fignocaption{width = 0.20\textwidth}{figures/qa.png}
  \vspace{-0.8cm}
  \begin{center}
    \textcolor{blue}{\bf \large hengxin0912@gmail.com}
  \end{center}
  \vspace{-0.5cm}
  \fignocaption{width = 0.55\textwidth}{figures/thankyou.jpg}
\end{frame}
%%%%%%%%%%%%%%%
\begin{frame}{未来工作}
  数据一致性问题的发展趋势:
  \begin{enumerate}
	\item give more consideration to SLA $\Rightarrow$ 应用价值观导向的数据一致性 (我们追随的发展趋势)
	\item poor definition of consistency $\Rightarrow$ 为弱一致性奠定理论基础
	\item poor understanding of boundaries $\Rightarrow$ 探索更强的数据一致性模型及理论界限
  \end{enumerate}
\end{frame}
