%%%%%%%%%%%%%%%%%%%%%%%%%%%%%%%%%%%%%%%%
\subsection{VPC: Pipelined-RAM 一致性验证}

\newcommand{\pram}{Pipelined-RAM}
\newcommand{\vpc}[1]{\ifthenelse{\isempty{#1}{}}{\textsf{VPC}}{\textsf{VPC-\MakeUppercase{#1}}}} 
\newcommand{\npc}{$\sf{NP}$-complete}
\newcommand{\npcn}{$\sf{NP}$-completeness}
\newcommand{\rwclosure}{\textsc{RW-Closure}}
\newcommand{\readcentric}{\textsc{Read-Centric}}
%%%%%%%%%%%%%%%
\begin{frame}{VPC 工作在技术框架中的位置}
  \fig{width = 0.50\textwidth}{figures/3d-framework-vpc.pdf}{VPC --- \pram{} 一致性验证.}
\end{frame}
%%%%%%%%%%%%%%%
\begin{frame}{VPC 问题定义}
  \begin{cdef}[VPC: Verifying PRAM Consistency]
    VPC 判定问题:
	\vspace{8pt}
    \begin{description}
	  \setlength{\itemsep}{8pt}
      \item[实例:] 系统执行 {\small (execution $e$; 即, 读写操作序列)}
      \item[问题:] 该执行是否满足 PRAM 一致性模型 {\small ($\mathcal{C}$)}? 
		\[
		  e \in \mathcal{C} \Rightarrow \set{0,1}?
		\]
    \end{description}    
  \end{cdef}

  \vspace{0.30cm}
  实例规模 $n$: 系统执行中操作的总数
\end{frame}
%%%%%%%%%%%%%%%
\begin{frame}{研究动机}
  \question{问题: 为什么要验证 Pipelined-RAM {\small (PRAM)} 一致性?}
  \vspace{0.10cm}

  \begin{description}
    \setlength{\itemsep}{10pt}
    \item[验证:] 用户与商家就数据一致性签订 SLA 协议 \\
	  \citeinbeamer{Amazon}{SOSP}{07} \citeinbeamer{Golab}{PODC}{11}
	  \vspace{2pt}
	  \begin{itemize}
		\item \textcolor{teal}{[商家]} 系统测试手段之一
		\item \textcolor{teal}{[用户]} 确认系统是否提供了其所声称的数据一致性 
	  \end{itemize}
	\pause
	\item[PRAM:] 存储系统常提供``会话'' {\small (session)} 一致性\\
      \citeinbeamer{Saito}{CSUR}{05} \citeinbeamer{Terry}{CACM}{13}
	  \vspace{2pt}
      \begin{itemize}
		\item 包含了弱一致性的诸多变体 
		\item 近似于 PRAM 一致性 \citeinbeamer{Brzezi$\acute{\text{n}}$ski}{PDP}{04} \citeinbeamer{Bailis}{VLDB}{13}
      \end{itemize}
  \end{description}
\end{frame}
%%%%%%%%%%%%%%%
\begin{frame}{VPC 问题分类}
  \begin{table}[!t]
    \centering
	\caption{VPC 问题的四种变体 (按``执行''的类型) 及复杂度分析 ($\textcolor{red}{[\ast]}$\textcolor{red}{: 本文工作}).}
	\renewcommand\arraystretch{1.2}
    \begin{tabular}{|c|c|c|}
      \hline
      & \it (S)ingle variable  & \it (M)ultiple variables
      \\ \hline
	  {\it write (D)uplicate values} &
	  \innercell{c}{VPC-SD \\ (\npc{}) $\textcolor{red}{[\ast]}$} &
	  \innercell{c}{VPC-MD \\ (\npc{}) $\textcolor{red}{[\ast]}$}
      \\ \hline
	  \only<1>{\it write (U)nique value}\only<2>{\cellcolor{brown!80}{\it write (U)nique value}} &
      \innercell{c}{VPC-SU \\ (P) \citeinbeamer{Golab}{PODC}{11}} &
      \innercell{c}{VPC-MU \\ (P) $\textcolor{red}{[\ast]}$}
      \\ \hline
    \end{tabular}
  \end{table}

  \vspace{10pt}
  
  \uncover<2->{\textcolor{brown}{\centerline{Read-mapping \citeinbeamer{Gibbons}{SICOMP}{97}: $\forall r, f(r) = w$.}}}
\end{frame}
%%%%%%%%%%%%%%%
\begin{frame}{VPC-SD (VPC-MD) 是 \npc{} 问题}
  \fig{width = 0.40\textwidth}{figures/vpcsd-npc.pdf}{对应于 \textsc{Unary 3-Partition} 实例 $A = \{2,2,1,1,1,1\}, m = 2, B = 4$ 的 VPC 执行.} 
\end{frame}
%%%%%%%%%%%%%%%
\begin{frame}{VPC-MU 的多项式算法 \rwclosure{}}
  \begin{figure}[h!]
    \centering
    \begin{adjustbox}{max totalsize = {0.65\textwidth}{1.00\textheight}, center}
	  % http://tex.stackexchange.com/questions/54794/using-a-pgfplots-style-legend-in-a-plain-old-tikzpicture#54834

% argument #1: any options
\newenvironment{customlegend}[1][]{%
    \begingroup
    % inits/clears the lists (which might be populated from previous
    % axes):
    \csname pgfplots@init@cleared@structures\endcsname
    \pgfplotsset{#1}%
}{%
    % draws the legend:
    \csname pgfplots@createlegend\endcsname
    \endgroup
}%

% makes \addlegendimage available (typically only available within an
% axis environment):
\def\addlegendimage{\csname pgfplots@addlegendimage\endcsname}

%%--------------------------------

% definition to insert numbers
\pgfkeys{/pgfplots/number in legend/.style={%
        /pgfplots/legend image code/.code={%
            \node at (0.295,-0.0225){#1};
        },%
    },
}
%%%%%%%%%%%%% For Legend %%%%%%%%%%%%%%%%%%%%%%

\begin{tikzpicture}[read/.style = {fill = orange, font = \Large}, 
write/.style = {fill = lightgray, font = \Large},
on grid, every node/.style = {node distance = 1.0cm and 1.6cm}, 
po/.style = {->, very thick, blue}]
%   	      \draw[step = {(1.5,1)}, style=help lines] (-1.5,0) grid (10.5,4);
	% process 0
	\begin{scope}[yshift = 4.0cm]
		\node (wy1) [write] at (0,0) {$Wy1$};
		\node (rf1) [read, right = of wy1] {$Rf1$};
		\node (rc1) [read, right = of rf1] {$Rc1$};
		\node (rz1) [read, right = of rc1] {$Rz1$};
		\node (ry1) [read, right = of rz1] {$Ry1$};
		\node (ra1) [read, right = of ry1] {$Ra1$};
		\node (rb1) [read, right = of ra1] {$Rb1$};
		\node (rx2) [read, right = of rb1] {$Rx2$};
	\end{scope}

	% process 1
	\begin{scope}
		\node (wf1) [write, node distance = 1.5cm, below = of rf1] {$Wf1$};
		\node (wx2) [write, node distance = 1.5cm and 3.0cm, below left = of rx2] {$Wx2$};
		\node (wc1) [write, node distance = 3.0cm, right = of wx2] {$Wc1$};
	\end{scope}

	% process 2
	\begin{scope}[]
		\node (wa1) [write, node distance = 3.0cm and 3.5cm, below left = of ra1] {$Wa1$};
		\node (wx3) [write, left = of wa1] {$Wx3$};
		\node (wz2) [write, left = of wx3] {$Wz2$};
		\node (wf2) [write, left = of wz2] {$Wf2$};
	\end{scope}

	% process 3
	\begin{scope}[]
		\node (wb1) [write, node distance = 4.5cm and 0.5cm, below right = of rb1] {$Wb1$};
		\node (wx5) [write, left = of wb1] {$Wx5$};
		\node (wy2) [write, left = of wx5] {$Wy2$};
		\node (wz1) [write, left = of wy2] {$Wz1$};
	\end{scope}

\begin{scope}[dotted, very thick, every node/.style = {font = \Large}]
  \node (lvnode) [node distance = 1.5cm, left = of wy1] {};
  \node (rvnode) [node distance = 1.0cm, right = of rx2] {};
  
  \node (p0) at (lvnode) {$p_0$};
  \draw (p0) -- (wy1);
  \node (rp0) at (rvnode) {};
  \draw (rx2) -- (rp0);

  \node (p1) [node distance = 1.5cm, below = of p0] {$p_1$};
  \draw (p1) -- (wf1);
  \node (rp1) [node distance = 1.5cm, below = of rp0] {};
  \draw (wc1) -- (rp1);

  \node (p2) [node distance = 1.5cm, below = of p1] {$p_2$};
  \draw (p2) -- (wf2);
  \node (rp2) [node distance = 1.5cm, below = of rp1] {};
  \draw (wa1) -- (rp2);

  \node (p3) [node distance = 1.5cm, below = of p2] {$p_3$};
  \draw (p3) -- (wz1);
  \node (rp3) [node distance = 1.5cm, below = of rp2] {};
  \draw (wb1) -- (rp3);
\end{scope}

\uncover<2->{
    % program order
    \begin{scope}[]
	  \draw[po] (wy1) -- (rf1);
	  \draw[po] (rf1) -- (rc1);	  
	  \draw[po] (rc1) -- (rz1);
	  \draw[po] (rz1) -- (ry1);
	  \draw[po] (ry1) -- (ra1);
	  \draw[po] (ra1) -- (rb1);
	  \draw[po] (rb1) -- (rx2);

	  \draw[po] (wf1) -- (wx2);
	  \draw[po] (wx2) -- (wc1);

	  \draw[po] (wf2) -- (wz2);
	  \draw[po] (wz2) -- (wx3);
	  \draw[po] (wx3) -- (wa1);

	  \draw[po] (wz1) -- (wy2);
	  \draw[po] (wy2) -- (wx5);
	  \draw[po] (wx5) -- (wb1);
	\end{scope}
  }

  \uncover<3->{
	% read-write mapping
	\begin{scope}[rwmap/.style={->, violet, very thick}]
		\draw [rwmap] (wf1) -- (rf1);
		\draw [rwmap] (wc1) [out = 160, in = -15] to (rc1.south);
		\draw [rwmap] (wz1) -- (rz1);
% 		\draw [rwmap] (wy1) [out = 20, in = 160] to (ry1);
		\draw [rwmap] (wy1) [out = -23, in = -157] to (ry1);
		\draw [rwmap] (wa1.north) -- (ra1.south);
		\draw [rwmap] (wb1) -- (rb1);
		\draw [rwmap] (wx2) -- (rx2.south);
	\end{scope}
  }

	% wprimew edge 
	\begin{scope}[wprimew/.style = {->, purple, dashed, ultra thick}]
	  \uncover<4->{\draw[wprimew] (wx3.north) to node [font = \Large, near start, color = black] {1} (wx2);}
	  \uncover<5->{\draw[wprimew] (wx5) to node [font = \Large, color = black] {2} (wx2);}

	  \uncover<6->{\draw[wprimew] (wz2.south) to [out = -45, in = 180] node [font = \Large, color = black] {3} (wz1);}
	  \uncover<7->{\draw[wprimew] (wf2.north) to node [font = \Large, color = black] {5} (wf1.west);}
	  \uncover<8>{\draw[wprimew] (wy2.north) to [out = 100, in = -20] node [font = \Large, very near start, color = black] {4} (wy1.south);}
	\end{scope}

\coordinate (legend-pos) at ($0.5*(p0) + 0.5*(rx2)$);

\begin{customlegend}[legend columns = -1, 
legend entries = { % <= in the following there are the entries
	$\; $ program order,
    $\; $ write-to order,
    $\; $ $w'wr$ order
},
legend style = {at = {($(legend-pos) + (0,1)$)}, anchor = south, /tikz/every even column/.append style = {column sep=0.5cm}, font = \Large}] % <= to define position and font legend
% the following are the "images" and numbers in the legend
    \addlegendimage{->, very thick, blue}
    \addlegendimage{->, violet, very thick}
    \addlegendimage{->, purple, dashed, ultra thick}
\end{customlegend}

\end{tikzpicture}
    \end{adjustbox}
	\caption{\rwclosure{} 算法示例: 在传递闭包之上迭代应用 $w'wr$ 规则.}
  \end{figure}

  TODO: 规则c
\end{frame}
%%%%%%%%%%%%%%%
\begin{frame}{VPC-MU 的多项式算法 \rwclosure{}}
  \begin{ctheorem}[\rwclosure{} 算法正确性]
	\vpc{mu} 实例满足 \pram{} 一致性当且仅当 \rwclosure{} 算法所得图是有向无环图.
  \end{ctheorem}

  \pause 
  TODO: proof

  \pause
  \centerline{\rwclosure{} 算法复杂度: $O(n^2) \cdot O(n^3) = O(n^5)$}
\end{frame}
%%%%%%%%%%%%%%%
\begin{frame}{VPC-MU 的多项式算法 \readcentric{}}
  关键点:

  \begin{ctheorem}[\rwclosure{} 算法正确性]
	\vpc{mu} 实例满足 \pram{} 一致性当且仅当 \readcentric{} 算法所得图是有向无环图.
  \end{ctheorem}

  \pause
  \centerline{\rwclosure{} 算法复杂度: }
\end{frame}
%%%%%%%%%%%%%%%
\begin{frame}{实验评估}
  
\end{frame}
%%%%%%%%%%%%%%%
\begin{frame}{VPC 的意义}
  \mdf{red}{blue}{}{teal}{
	\begin{enumerate}
	  \item \readcentric{} 算法可用于测试系统是否正确实现了 \pram{} 一致性模型
	  \item \npcn{} 结果有助于理解弱一致性模型的复杂度
	\end{enumerate}
  }
\end{frame}
%%%%%%%%%%%%%%%
