\section{研究背景}

%%%%%%%%%%%%%%%
\begin{frame}{分布式应用}
  开放互联的网络环境下, 网络应用 (web applications) 分布部署.

  \todo{图: 分布式应用 (Weibo social network 举例)}
\end{frame}
%%%%%%%%%%%%%%%
\begin{frame}{分布式应用}
  \begin{center}
    应用三层架构: 表示层, 业务层, \textcolor{red}{数据层}
  \end{center}

  \todo{图: 三层架构}

  \begin{columns}[t]
    \column{0.50\textwidth}
      数据层中间件:
      \begin{itemize}
	\item 屏蔽底层数据具体形态
	\item 简化业务层开发
      \end{itemize}
    \column{0.50\textwidth}
    数据层中间件\mathbfblue{H^3L}特性:
      \begin{itemize}
	\item high availability
	\item high fault-tolerance
	\item high scalability
	\item low latency
      \end{itemize}
  \end{columns}

  \only<2->{
    \begin{center}
      数据形态: 共享数据 (shared data) vs. 分布数据 (distributed data)
    \end{center}
  }
\end{frame}
%%%%%%%%%%%%%%%
\begin{frame}{数据层: 共享数据}
  \fig{width = 0.35\textwidth}{figures/shared-data.png}{(集中式)共享数据系统 \todo{重绘}.}

  \begin{itemize}
    \item 单点故障 $\nRightarrow$ high availability
    \item 性能瓶颈 $\nRightarrow$ low latency
    \item 超荷负载 $\nRightarrow$ high scalability
  \end{itemize}
\end{frame}
%%%%%%%%%%%%%%%
\begin{frame}{数据层: 分布数据}
  \fig{width = 0.40\textwidth}{figures/distributed-data.pdf}{分布数据系统 \todo{动画: partition+replication}.}

  \begin{center}
    {\LARGE distributed data : partition + replication}
  \end{center}

  % \begin{itemize}
  %   \item 容灾备份 $\Rightarrow$ \mathbfblue{A}vailability
  %   \item 就近访问 $\Rightarrow$ low \mathbfblue{L}atency
  %   \item 负载均衡 $\Rightarrow$ high \mathbfblue{S}calability
  % \end{itemize}
\end{frame}
%%%%%%%%%%%%%%%
\begin{frame}{分布数据应用举例 (I)}
  \fig{width = 0.618\textwidth}{figures/dsss.pdf}
  {分布式存储系统 (\textcolor{blue}{\scriptsize 开源 [左]} \& \textcolor{red}{\scriptsize 商用 [右]}).}

  应用需求 \citeinbeamer{Facebook}{OSDI}{10} vs. ``分布数据'':
  \begin{description}
    \item[低延迟:] 就近访问副本数据
    \item[高可用性, 高容错性:] 备份容灾 
  \end{description}
\end{frame}
%%%%%%%%%%%%%%%
\begin{frame}{分布数据应用举例 (II)}
  \fig{width = 0.85\textwidth}{figures/file-share.pdf}
  {个人多设备文件共享 {(\textcolor{blue}{\scriptsize [基于云] C/S 结构 [左]} \& 
  \textcolor{red}{\scriptsize P2P 结构 [右]}).}}

  应用需求与特点 \citeinbeamer{Strauss}{MIT Thesis}{10} vs. ``分布数据'':
  \begin{description}
    \item[功能需求:] 文件副本
    \item[网络断连:] 备份容灾; 离线可用
  \end{description}
\end{frame}
%%%%%%%%%%%%%%%
% \begin{frame}{分布副本数据的典型应用 (三)}
%   \fig{width = 0.80\textwidth}{figures/coordination.pdf}
%   {分布式协同应用(上)及服务(下).}
% 
%   分布式协同服务需求 \citeinbeamer{Burrows}{OSDI}{06} vs. ``副本技术'':
%   \begin{description}
%     \item[高性能:] 低延迟就近``读''副本 \citeinbeamer{Yahoo!}{USENIXATC}{10}
%     \item[高可靠性:] 避免单点协同故障
%   \end{description}
% \end{frame}
%%%%%%%%%%%%%%%
% \begin{frame}{数据的形态 (总结)}
%   分布共享数据常以\question{副本}的形态存在:
%   \begin{description}
%     \item[功能需求:] 如, 多设备文件共享
%     \item[系统性能:] 低延迟就近访问
%     \item[高可靠性:] 备份容灾; 避免单点故障
%   \end{description}
% \end{frame}
%%%%%%%%%%%%%%%
