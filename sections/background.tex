\section{研究背景}

%%%%%%%%%%%%%%%
\begin{frame}{分布式应用}
  \fignocaption{width = 0.30\textwidth}{figures/sina-weibo.png}

  \todo{动画: 分布部署}

\end{frame}
%%%%%%%%%%%%%%%
\begin{frame}{分布数据}
  (\todo{动画: partition + replication})

  \fignocaption{width = 0.45\textwidth}{figures/riak-data-distribution.png}

  \begin{center}
    {\Large distributed data : partition \& replication}
  \end{center}

  % \begin{itemize}
  %   \item 容灾备份 $\Rightarrow$ \mathbfblue{A}vailability
  %   \item 就近访问 $\Rightarrow$ low \mathbfblue{L}atency
  %   \item 负载均衡 $\Rightarrow$ high \mathbfblue{S}calability
  % \end{itemize}
\end{frame}
%%%%%%%%%%%%%%%
\begin{frame}{分布数据典型应用 (I)}
  \fig{width = 0.55\textwidth}{figures/dsss.pdf}
  {分布式存储系统 (\textcolor{blue}{\scriptsize 开源 [左]} \& \textcolor{red}{\scriptsize 商用 [右]}).}

  \begin{description}
    \item[低延迟:] 就近访问副本数据
    \item[高可用性, 高容错性:] 备份容灾 
  \end{description}
\end{frame}
%%%%%%%%%%%%%%%
\begin{frame}{分布数据典型应用 (II)}
  \fig{width = 0.75\textwidth}{figures/file-share.pdf}
  {个人多设备文件共享 {(\textcolor{blue}{\scriptsize [基于云] C/S 结构 [左]} \& 
  \textcolor{red}{\scriptsize P2P 结构 [右]}).}}

  \begin{description}
    \item[功能需求:] 文件副本 \citeinbeamer{Strauss}{MIT Thesis}{10}
    \item[网络断连:] 备份容灾; 离线可用
  \end{description}
\end{frame}
%%%%%%%%%%%%%%%
% \begin{frame}{分布副本数据的典型应用 (三)}
%   \fig{width = 0.80\textwidth}{figures/coordination.pdf}
%   {分布式协同应用(上)及服务(下).}
% 
%   分布式协同服务需求 \citeinbeamer{Burrows}{OSDI}{06} vs. ``副本技术'':
%   \begin{description}
%     \item[高性能:] 低延迟就近``读''副本 \citeinbeamer{Yahoo!}{USENIXATC}{10}
%     \item[高可靠性:] 避免单点协同故障
%   \end{description}
% \end{frame}
%%%%%%%%%%%%%%%
\begin{frame}{分布式应用访问分布数据}
  \fignocaption{width = 0.40\textwidth}{figures/distributed-data.pdf}
  
  \todo{重绘: 显示 partition \& replication}
\end{frame}
%%%%%%%%%%%%%%%
