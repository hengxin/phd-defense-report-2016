\section{研究背景}

\begin{frame}{分布式应用}
  开放互联的网络环境, 软件/应用体现出``分布性'': 地理分布部署

  \todo{图: 分布式应用}
\end{frame}

%%%%%%%%%%%%%%%
\begin{frame}{数据复制}
  \begin{center}
    数据共享
  \end{center}

  \todo{图示: 集中式 $\Rightarrow$ 数据复制}

  \begin{columns}
    \column{0.50\textwidth}
      集中式数据:
      \begin{itemize}
	\item 性能瓶颈
	\item 单点故障
      \end{itemize}
    \column{0.50\textwidth}
      数据复制:
      \begin{itemize}
	\item 负载均衡, 就近访问
	\item 容灾备份
      \end{itemize}
  \end{columns}
\end{frame}
%%%%%%%%%%%%%%%
\begin{frame}{分布副本数据的典型应用 (一)}
  \fig{width = 0.618\textwidth}{figures/dsss.pdf}
  {分布式存储系统 (\textcolor{blue}{\scriptsize 开源 [左]} \& \textcolor{red}{\scriptsize 商用 [右]})}.

  上层业务需求 \citeinbeamer{Facebook}{OSDI}{10} vs. ``副本技术'':
  \begin{description}
    \item[高性能:] 低延迟就近访问副本数据
    \item[高可靠性:] 备份容灾; 随时可用
  \end{description}
\end{frame}
%%%%%%%%%%%%%%%
\begin{frame}{分布数据的典型应用 (二)}
  \fig{width = 0.85\textwidth}{figures/file-share.pdf}
  {个人多设备文件共享 {(\textcolor{blue}{\scriptsize [基于云] C/S 结构 [左]} \& 
  \textcolor{red}{\scriptsize P2P 结构 [右]})}}

  应用需求与特点 \citeinbeamer{Strauss}{MIT Thesis}{10} vs. ``副本技术'':
  \begin{description}
    \item[功能需求:] 文件副本
    \item[网络断连:] 容灾备份; 离线可用
  \end{description}
\end{frame}
%%%%%%%%%%%%%%%
% \begin{frame}{分布副本数据的典型应用 (三)}
%   \fig{width = 0.80\textwidth}{figures/coordination.pdf}
%   {分布式协同应用(上)及服务(下).}
% 
%   分布式协同服务需求 \citeinbeamer{Burrows}{OSDI}{06} vs. ``副本技术'':
%   \begin{description}
%     \item[高性能:] 低延迟就近``读''副本 \citeinbeamer{Yahoo!}{USENIXATC}{10}
%     \item[高可靠性:] 避免单点协同故障
%   \end{description}
% \end{frame}
%%%%%%%%%%%%%%%
\begin{frame}{数据的形态 (总结)}
  分布共享数据常以\question{副本}的形态存在:
  \begin{description}
    \item[功能需求:] 如, 多设备文件共享
    \item[系统性能:] 低延迟就近访问
    \item[高可靠性:] 备份容灾; 避免单点故障
  \end{description}
\end{frame}
