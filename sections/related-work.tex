\section{相关工作}	\label{section:related-work}

%%%%%%%%%%%%%%%
\begin{frame}{相关工作分类}
  
\end{frame}
%%%%%%%%%%%%%%%
\begin{frame}{\ideadt{}的研究理念 (一)}
  \ideadt{}的\textcolor{red}{读写寄存器}一致性模型 \textcolor{blue}{\footnotesize (多处理器系统)}:
  \pause
  \vspace{8pt}
  \begin{description}
	\setlength{\itemsep}{5pt}
	\item[典型:] Hybrid consistency \citeinbeamer{Attiya}{SIAM J. Comput.}{98}
	\item[思想:] 将操作分为强弱两类
	  \pause
	\item[其它:] ``带同步的''一致性模型 \citeinbeamer{Dubois}{IEEE Computer}{88} \citeinbeamer{Steinke}{JACM}{04}
	\item[特点:] 强调正确性 {\footnotesize (properly synchronized)}
	\item[总评:] 相关工作丰富; 理论扎实
  \end{description}
\end{frame}
%%%%%%%%%%%%%%%
\begin{frame}{\ideadt{}的研究理念 (二)}
  \ideadt{}的\textcolor{red}{读写寄存器}一致性模型 \textcolor{blue}{\footnotesize (分布式系统)}:
  \vspace{8pt}
  \begin{description}
	\setlength{\itemsep}{5pt}
	\item[思想:] 借鉴并发展 Hybrid consistency 的思想
	\item[典型:] 
	  \begin{itemize}
		\item Causal/immediate/forced consistency \citeinbeamer{Ladin}{TOCS}{92}
		\item RedBlue consistency \citeinbeamer{Li}{OSDI}{12}
		\item Apache Cassandra~\footnote{\url{http://cassandra.apache.org/}} \citeinbeamer{Facebook}{SIGOPS OSR}{10}
		\item Pileus \citeinbeamer{Terry}{SOSP}{13}
	  \end{itemize}
	  \pause
	\item[特点:] 更细粒度的多一致性模型共存、更能容忍数据不一致
	\item[总评:] 已成趋势; 缺少基础理论工作
  \end{description}
\end{frame}
%%%%%%%%%%%%%%%
\begin{frame}{\ideadt{}的研究理念 (三)}
  \ideadt{}的\textcolor{red}{事务}一致性模型 \textcolor{blue}{\footnotesize (分布式系统)}:
\end{frame}
%%%%%%%%%%%%%%%
\begin{frame}{\idearm{}的研究理念}
  \begin{table}
	\centering
	\begin{tabular}{|c|c|c|c|c|}
	  \hline
	  & 读写寄存器 & 事务
	\end{tabular}
	\caption{相关工作二: \idearm{}的研究理念}
	\label{tbl:idearm-related-work}
  \end{table}
\end{frame}
%%%%%%%%%%%%%%%
%%%%%%%%%%%%%%%
