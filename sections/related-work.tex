\section{相关工作}	\label{section:related-work}

%%%%%%%%%%%%%%%
\begin{frame}{相关工作分类}
  \begin{table}[]
	\centering
	\caption{\idea{}研究理念相关工作.}
	\label{tbl:related-work-categories}
	\renewcommand\arraystretch{2}
	\resizebox{\textwidth}{!}{%
	  \begin{tabular}{cc|c|c|c|c|}
		\cline{3-6}
		\multicolumn{2}{c|}{} & \multicolumn{2}{c|}{\textbf{读写寄存器}} & \multicolumn{2}{c|}{\textbf{事务}} \\ \cline{3-6}
		\multicolumn{2}{c|}{} & 多处理器系统 & 分布式系统 & 多处理器系统 & 分布式系统 \\ \hline
		\multicolumn{2}{|c|}{\textbf{\ideadt{}}} &  &  &  &  \\ \hline
		\multicolumn{1}{|c|}{\multirow{2}{*}{\textbf{\idearm{}}}} & 验证 &  &  &  &  \\ \cline{2-6} 
		\multicolumn{1}{|c|}{} & 量化 &  &  &  &  \\ \hline
	  \end{tabular}
    }
  \end{table}
\end{frame}
%%%%%%%%%%%%%%%
\begin{frame}{\ideadt{}的研究理念 (一)}
  \begin{table}[]
	\centering
	\renewcommand\arraystretch{1.5}
	\resizebox{\textwidth}{!}{%
	  \begin{tabular}{cc|c|c|c|c|}
		\cline{3-6}
		\multicolumn{2}{c|}{} & \multicolumn{2}{c|}{\textbf{读写寄存器}} & \multicolumn{2}{c|}{\textbf{事务}} \\ \cline{3-6} 
		\multicolumn{2}{c|}{} & 多处理器系统 & 分布式系统 & 多处理器系统 & 分布式系统 \\ \hline
		\multicolumn{2}{|c|}{\textbf{\ideadt{}}} & \only<1-3>{\textcolor{red}{\bf \checkmark}} 
		\only<4->{\begin{tabular}[c]{@{}c@{}}\textcolor{red}{相关工作丰富}\\ \textcolor{red}{理论扎实}\end{tabular}} &  &  &  \\ \hline
	  \end{tabular}%
	}
  \end{table}

  \pause 

  \begin{description}
	\setlength{\itemsep}{5pt}
	\item[典型:] Hybrid consistency \citeinbeamer{Attiya}{SIAM J. Comput.}{98}
	\item[思想:] 将操作分为强弱两类
	  \pause
	\item[其它:] ``带同步的''一致性模型 \citeinbeamer{Dubois}{IEEE Computer}{88} \citeinbeamer{Steinke}{JACM}{04}
	\item[特点:] 强调正确性 {\footnotesize (properly synchronized)}
  \end{description}
\end{frame}
%%%%%%%%%%%%%%%
\begin{frame}{\ideadt{}的研究理念 (二)}
  \begin{table}[]
	\centering
	\renewcommand\arraystretch{1.5}
	\resizebox{\textwidth}{!}{%
	  \begin{tabular}{cc|c|c|c|c|}
		\cline{3-6}
		\multicolumn{2}{c|}{} & \multicolumn{2}{c|}{\textbf{读写寄存器}} & \multicolumn{2}{c|}{\textbf{事务}} \\ \cline{3-6} 
		\multicolumn{2}{c|}{} & 多处理器系统 & 分布式系统 & 多处理器系统 & 分布式系统 \\ \hline
		\multicolumn{2}{|c|}{\textbf{\ideadt{}}} & {\begin{tabular}[c]{@{}c@{}}{\small 相关工作丰富}\\ {\small 理论扎实}\end{tabular}}
		& \only<1-3>{\textcolor{red}{\bf \checkmark}} 
		\only<4->{\begin{tabular}[c]{@{}c@{}}\textcolor{red}{渐成趋势}\\ \textcolor{red}{理论欠缺}\end{tabular}} &  &  \\ \hline
	  \end{tabular}%
	}
  \end{table}

  \pause 

  \begin{description}
	\setlength{\itemsep}{5pt}
	\item[思想:] 借鉴并发展 Hybrid consistency 的思想
	\item[典型:] 
	  \begin{itemize}
		\item Causal+forced+immediate operations \citeinbeamer{Ladin}{TOCS}{92}
		\item RedBlue consistency \citeinbeamer{Li}{OSDI}{12}
		% \item Apache Cassandra~\footnote{\url{http://cassandra.apache.org/}} \citeinbeamer{Facebook}{SIGOPS OSR}{10}
		\item Pileus \citeinbeamer{Terry}{SOSP}{13}
	  \end{itemize}
	  \pause
	\item[特点:] 更细粒度的多一致性模型共存、更能容忍数据不一致
  \end{description}
\end{frame}
%%%%%%%%%%%%%%%
\begin{frame}{\ideadt{}的研究理念 (三)}
  \begin{table}[]
	\centering
	\renewcommand\arraystretch{1.5}
	\resizebox{\textwidth}{!}{%
	  \begin{tabular}{cc|c|c|c|c|}
		\cline{3-6}
		\multicolumn{2}{c|}{} & \multicolumn{2}{c|}{\textbf{读写寄存器}} & \multicolumn{2}{c|}{\textbf{事务}} \\ \cline{3-6} 
		\multicolumn{2}{c|}{} & 多处理器系统 & 分布式系统 & 多处理器系统 & 分布式系统 \\ \hline
		\multicolumn{2}{|c|}{\textbf{\ideadt{}}} & {\begin{tabular}[c]{@{}c@{}}{\small 相关工作丰富}\\ {\small 理论扎实}\end{tabular}}
		& \begin{tabular}[c]{@{}c@{}}{\small 渐成趋势}\\ {\small 理论欠缺}\end{tabular} 
		& \only<1>{\textcolor{red}{\small 软件事务内存}}\only<2->{\textcolor{gray}{\small 软件事务内存}}
		& \only<2-3>{\textcolor{red}{\bf \checkmark}} 
		  \only<4->{\textcolor{red}{探索阶段}} \\ \hline
		  % \only<4->{\begin{tabular}[c]{@{}c@{}}\textcolor{red}{探索阶段}\\ \textcolor{red}{系统、理论欠缺}\end{tabular}} \\ \hline
	  \end{tabular}
	}
  \end{table}

  \pause

  \begin{description}
	\setlength{\itemsep}{5pt}
	\item[思想:] 多个事务一致性模型共存
	\item[典型:] 
	  \begin{itemize}
		\item RC-SR {\scriptsize (relaxed currency serializability)} \citeinbeamer{Bernstein}{SIGMOD}{06}
		\item Pileus consistency choices \citeinbeamer{Terry}{MSR-TR}{13}
		\item Multi-level CSI {\scriptsize (Causal Snapshot Isolation)} \citeinbeamer{Tripathi}{BigData}{15}
	  \end{itemize}
	  \pause
	\item[挑战:] ``多样化''事务语义; 可扩展的系统实现
  \end{description}
\end{frame}
%%%%%%%%%%%%%%%
\begin{frame}{\idearm{}的研究理念 (一)}
  \begin{table}
	\centering
	\begin{tabular}{|c|c|c|c|c|}
	  \hline
	  & 读写寄存器 & 事务
	\end{tabular}
	\caption{相关工作二: \idearm{}的研究理念}
	\label{tbl:idearm-related-work}
  \end{table}
\end{frame}
%%%%%%%%%%%%%%%
%%%%%%%%%%%%%%%
