\section{技术框架}

\begin{comment}
%%%%%%%%%%%%%%%%%%%%%%%%%%%%%%
\subsection{理论模型:分布共享数据服务}

%%%%%%%%%%%%%%%
\begin{frame}{分布共享数据服务}
  \begin{columns}
	\column{0.50\textwidth}
	  \fignocaption{width = 0.60\textwidth}{figures/shared-data-clients.pdf}
	  \begin{center}
		共享数据系统
	  \end{center}
	\pause
	\column{0.50\textwidth}
	  \fignocaption{width = 0.70\textwidth}{figures/distributed-data-clients.pdf}
	  \begin{center}
		分布数据系统
	  \end{center}
  \end{columns}
\end{frame}
%%%%%%%%%%%%%%%
\begin{frame}{分布共享数据服务}
  \fignocaption{width = 0.45\textwidth}{figures/distributed-shared-data-clients.pdf}
  \begin{center}
	分布共享数据服务作为中间件管理分布数据
  \end{center}
\end{frame}
%%%%%%%%%%%%%%%
\begin{frame}{分布共享数据服务}
  \fignocaption{width = 0.35\textwidth}{figures/dsds.pdf}
  \begin{center}
	\textcolor{blue!80}{分布共享数据服务 \term{中间件}:} 在分布数据之上提供共享数据的抽象
  \end{center}
\end{frame}
%%%%%%%%%%%%%%%
\begin{frame}{分布共享数据服务 (注)}
  \begin{center}
	分布共享内存模型 (多处理器系统)\\
	\textcolor{blue}{[传统概念]}\\
	\vspace{0.10cm} +\\ \vspace{0.10cm}
	分布数据系统\\
	\textcolor{blue}{[新平台]}
	
	\uncover<2->{
	  \vspace{0.3cm}
	\color{red}\rule{0.618\linewidth}{3pt}
	  \vspace{0.2cm}

	\begin{center}
	  \textcolor{blue}{新平台凸显应用价值观:}
	  \begin{enumerate}
		\centering
		\item 多样化, 可调节
		\item 精细化, 可度量
	  \end{enumerate}
	\end{center}
	}
  \end{center}
\end{frame}
%%%%%%%%%%%%%%%
%%%%%%%%%%%%%%%%%%%%%%%%%%%%%%
\subsection{技术框架}
\end{comment}

%%%%%%%%%%%%%%%
\begin{frame}{分布数据一致性问题}
  \fignocaption{width = 0.618\textwidth}{figures/distributed-data-consistency-explained.pdf}
\end{frame}

% 分布数据一致性问题:
% \begin{itemize}
%   \item [\textcolor{blue}{\bf \cmark}] 分布: 分区 + 副本
%   \item [\textcolor{red}{\bf \xmark}] 数据: 数据类型
%   \item [\textcolor{red}{\bf \xmark}] 一致性: 关键问题
% \end{itemize}

% \textcolor{blue}{数据类型:}
% \begin{itemize}
%   \item 读写寄存器/变量 \textcolor{cyan}{\small ($x,y$)}
%   \item 数据结构 \textcolor{cyan}{\small (\textsc{Set, List})}
%   \item 事务 \textcolor{cyan}{\small (\textsc{Tx})}
% \end{itemize}

% \textcolor{blue}{一致性关键问题:}
% \begin{itemize}
%   \item 模型 \textcolor{cyan}{\small (semantics; 是什么)}
%   \item 机制 \textcolor{cyan}{\small (mechanism; 怎么做)}
%   \item 度量 \textcolor{cyan}{\small (measurement; 怎么样)}
% \end{itemize}
%%%%%%%%%%%%%%%
\begin{frame}{技术框架}
  \fig{width = 0.50\textwidth}{figures/3d-framework.pdf}{``一个基础, 三个维度''技术框架.}
\end{frame}
%%%%%%%%%%%%%%%
\begin{frame}{基础: 数据类型}
	数据类型: 从个体到群组
	\pause
	\vspace{0.20cm}
	\begin{itemize}
	  \setlength{\itemsep}{8pt}
	  \item<2-> 读写寄存器
	  \item<3-> 事务对象
		\begin{itemize}
		  \setlength{\itemsep}{4pt}
		  \item 事务 $\triangleq$ 多个读写寄存器的操作序列
		  \item 支持 ``all-or-none'' 语义
		  \item 易于开发并发应用
		\end{itemize}
	\end{itemize}
\end{frame}
%%%%%%%%%%%%%%%
\begin{frame}{维度一: 一致性模型}
  一致性模型 \citeinbeamer{Steinke}{JACM}{04} \citeinbeamer{Adya}{Thesis}{99}:
  \vspace{0.20cm}
  \begin{itemize}
    \item 多进程并发操作某数据类型
	\item 规定各操作的语义
	  % \begin{itemize}
	  %   \setlength{\itemsep}{2pt}
	  %   \item 读写变量: 读操作允许的返回值 
	  %   \item 事务对象: 事务创建与提交操作的语义
	  % \end{itemize}
  \end{itemize}

  \vspace{0.20cm}
  \fig{width = 0.50\textwidth}{figures/consistency-model-tree.png}{一致性模型强弱关系 \scriptsize {(来自 \citeinbeamer{Bailis}{VLDB}{14})}.}
\end{frame}
%%%%%%%%%%%%%%%
\begin{frame}{维度一: 一致性模型}
  一致性模型的\textcolor{blue}{集合}定义:
  \begin{align*}
	\uncover<2->{\textrm{系统执行 } \; e &\triangleq \textrm{该执行所产生的事件的序列} \\[5pt]}
	\uncover<3->{\textrm{分布式系统 } \; \mathcal{S} &\triangleq \set{\textrm{该系统的所有可能执行}} \\[5pt]}
	\uncover<4->{\textrm{一致性模型 } \; \mathcal{C} &\triangleq \set{\textrm{该模型所允许的所有系统执行}}}
  \end{align*}

  \only<5->{\fignocaption{width = 0.20\textwidth}{figures/specs.png}}
\end{frame}
%%%%%%%%%%%%%%%
\begin{frame}{维度二: 一致性实现机制}
  \centerline{给定一致性模型 $\mathcal{C}$, 设计系统$\mathcal{S}$:}
  \[
	\forall e \in \mathcal{S}: e \in \mathcal{C}.
  \]
  \[
	\textrm{\it i.e., } \mathcal{S} \subseteq \mathcal{C}.
  \]

  \pause
  \vspace{0.50cm}

  \begin{center}
	``多样化, 可调节''研究理念的挑战:
	\vspace{8pt}
	\begin{itemize}
	  \centering
	  \item 兼容的混合一致性模型
	  \item 实现手段之一: 参数化
	\end{itemize}
  \end{center}
\end{frame}
%%%%%%%%%%%%%%%
\begin{frame}{维度三: 一致性度量}
  \begin{columns}
	\column{0.60\textwidth}
	  \begin{center}
		给定系统$\mathcal{S}$及一致性模型$\mathcal{C}$,
	  \end{center}

	  \pause

	  \begin{center}
		对于$e \in \mathcal{S}$:
		\begin{description}
		  \centering
		  \setlength{\itemsep}{5pt}
		  \item[验证 (verify):] $e \in \mathcal{C} \textcolor{red}{\mathbf{?}} \Rightarrow \set{0,1}$ 
		  \item[量化 (quantify):] $e \in \mathcal{C} \textcolor{red}{\mathbf{?}} \Rightarrow (0,1)$ 
		\end{description}
	  \end{center}
	\column{0.40\textwidth}
	  \fignocaption{width = 0.50\textwidth}{figures/qos-radio.png}
  \end{columns}

  \pause
  \vspace{0.60cm}

  \begin{center}
	``精细化, 可度量''研究理念的挑战:
	\vspace{8pt}
	\begin{description}
	  \centering
	  \item[验证:] 算法设计、复杂度分析
	  \item[量化:] 数学建模与求解
	\end{description}
  \end{center}
\end{frame}
%%%%%%%%%%%%%%%
