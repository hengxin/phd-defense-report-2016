\section{技术框架}

\begin{comment}
%%%%%%%%%%%%%%%%%%%%%%%%%%%%%%
\subsection{理论模型:分布共享数据服务}

%%%%%%%%%%%%%%%
\begin{frame}{分布共享数据服务}
  \begin{columns}
	\column{0.50\textwidth}
	  \fignocaption{width = 0.60\textwidth}{figures/shared-data-clients.pdf}
	  \begin{center}
		共享数据系统
	  \end{center}
	\pause
	\column{0.50\textwidth}
	  \fignocaption{width = 0.70\textwidth}{figures/distributed-data-clients.pdf}
	  \begin{center}
		分布数据系统
	  \end{center}
  \end{columns}
\end{frame}
%%%%%%%%%%%%%%%
\begin{frame}{分布共享数据服务}
  \fignocaption{width = 0.45\textwidth}{figures/distributed-shared-data-clients.pdf}
  \begin{center}
	分布共享数据服务作为中间件管理分布数据
  \end{center}
\end{frame}
%%%%%%%%%%%%%%%
\begin{frame}{分布共享数据服务}
  \fignocaption{width = 0.35\textwidth}{figures/dsds.pdf}
  \begin{center}
	\textcolor{blue!80}{分布共享数据服务 \term{中间件}:} 在分布数据之上提供共享数据的抽象
  \end{center}
\end{frame}
%%%%%%%%%%%%%%%
\begin{frame}{分布共享数据服务 (注)}
  \begin{center}
	分布共享内存模型 (多处理器系统)\\
	\textcolor{blue}{[传统概念]}\\
	\vspace{0.10cm} +\\ \vspace{0.10cm}
	分布数据系统\\
	\textcolor{blue}{[新平台]}
	
	\uncover<2->{
	  \vspace{0.3cm}
	\color{red}\rule{0.618\linewidth}{3pt}
	  \vspace{0.2cm}

	\begin{center}
	  \textcolor{blue}{新平台凸显应用价值观:}
	  \begin{enumerate}
		\centering
		\item 多样化, 可调节
		\item 精细化, 可度量
	  \end{enumerate}
	\end{center}
	}
  \end{center}
\end{frame}
%%%%%%%%%%%%%%%
%%%%%%%%%%%%%%%%%%%%%%%%%%%%%%
\subsection{技术框架}
\end{comment}

%%%%%%%%%%%%%%%
\begin{frame}{分布数据一致性问题}
  \fignocaption{width = 0.618\textwidth}{figures/distributed-data-consistency-explained.pdf}
\end{frame}

% 分布数据一致性问题:
% \begin{itemize}
%   \item [\textcolor{blue}{\bf \cmark}] 分布: 分区 + 副本
%   \item [\textcolor{red}{\bf \xmark}] 数据: 数据类型
%   \item [\textcolor{red}{\bf \xmark}] 一致性: 关键问题
% \end{itemize}

% \textcolor{blue}{数据类型:}
% \begin{itemize}
%   \item 读写寄存器/变量 \textcolor{cyan}{\small ($x,y$)}
%   \item 数据结构 \textcolor{cyan}{\small (\textsc{Set, List})}
%   \item 事务 \textcolor{cyan}{\small (\textsc{Tx})}
% \end{itemize}

% \textcolor{blue}{一致性关键问题:}
% \begin{itemize}
%   \item 模型 \textcolor{cyan}{\small (semantics; 是什么)}
%   \item 机制 \textcolor{cyan}{\small (mechanism; 怎么做)}
%   \item 度量 \textcolor{cyan}{\small (measurement; 怎么样)}
% \end{itemize}
%%%%%%%%%%%%%%%
\begin{frame}{技术框架}
  \fig{width = 0.50\textwidth}{figures/3d-framework.pdf}{``一个基础, 三个维度''技术框架.}
  % \begin{figure}[h!]
  %   \centering
  %   \begin{adjustbox}{max totalsize = {0.70\textwidth}{1.00\textheight}, center}
  %     \tdplotsetmaincoords{60}{135}

\begin{tikzpicture}[font = \scriptsize, framed, background rectangle/.style = {draw = teal}, scale = 2.6, tdplot_main_coords]
\coordinate (O) at (0,0,0);
\coordinate (x) at (1,0,0);
\coordinate (y) at (0,1,0);
\coordinate (z) at (0,0,1);

\uncover<1->{
\node[font = \normalsize, rotate = -90, teal] at (0,1.4,1.4) {\textsc{Datatype}};
}

\uncover<2->{
  \draw[thick, >=Stealth, ->] (O) to node[sloped, below, near end, align = center] {\texttt{\rmfamily Mechanism}\\{\tiny (\texttt{机制: 怎么做})}} (x);
  \draw[thick, >=Stealth, ->] (O) to node[anchor = north, sloped, below, near end, align = center] {\texttt{\rmfamily Measurement}\\{\tiny (\texttt{度量: 怎么样})}} (y);
  \draw[thick, >=Stealth, ->] (O) to (z) node[anchor = south, align = center] {\texttt{\rmfamily Semantics}\\{\tiny (\texttt{模型: 是什么})}};
}

\coordinate (xz) at (1,0,1);
\coordinate (yz) at (0,1,1);

\uncover<3->{
\draw[canvas is xz plane at y = 0, transform shape, draw = blue, fill = blue!50, opacity = 0.5] (xz) rectangle (O);
\node[canvas is xz plane at y = 0, align = center] at (0.5,0,0.5) {\reflectbox{\texttt{多样化},} \\\reflectbox{\texttt{可调节}}};
}

\uncover<4->{
\draw[canvas is yz plane at x = 0, transform shape, draw = brown, fill = brown!50, opacity = 0.5] (yz) rectangle (O);
\node[canvas is yz plane at x = 0, align = center] at (0,0.5,0.5) {\texttt{精细化},\\\texttt{可度量}};
}
\end{tikzpicture}
  %   \end{adjustbox}
  % \end{figure}
\end{frame}
%%%%%%%%%%%%%%%
\begin{frame}{基础: 数据类型}
	\begin{enumerate}
	  \setlength{\itemsep}{8pt}
	  \item 读写寄存器
	  \item 事务对象
		\begin{itemize}
		  \setlength{\itemsep}{4pt}
		  \item 事务 $\triangleq$ 多个读写寄存器的操作序列
		  \item 支持 ``all-or-none'' 语义
		  \item 易于开发并发应用
		\end{itemize}
	\end{enumerate}
\end{frame}
%%%%%%%%%%%%%%%
\begin{frame}{维度一: 一致性模型}
  \begin{columns}
	\column{0.55\textwidth}
	  一致性模型 \citeinbeamer{Steinke}{JACM}{04} \citeinbeamer{Adya}{Thesis}{99}:
	  \vspace{0.20cm}
	  \begin{itemize}
		\item 多进程并发操作某数据类型
		\item 规定各操作的语义 {\small (合法返回值)}
	  \end{itemize}
	\column{0.40\textwidth}
	  \uncover<2->{\fignocaption{width = 0.45\textwidth}{figures/specs.png}}
  \end{columns}

  \fig{width = 0.45\textwidth}{figures/consistency-model-tree.png}{一致性模型强弱关系 \scriptsize {(来自 \citeinbeamer{Bailis}{VLDB}{14})}.}
\end{frame}
%%%%%%%%%%%%%%%
% \begin{frame}{维度一: 一致性模型}
%   一致性模型的\textcolor{blue}{集合}定义:
%   \begin{align*}
% 	\uncover<2->{\textrm{系统执行 } \; e &\triangleq \textrm{该执行所产生的事件的序列} \\[5pt]}
% 	\uncover<3->{\textrm{分布式系统 } \; \mathcal{S} &\triangleq \set{\textrm{该系统的所有可能执行}} \\[5pt]}
% 	\uncover<4->{\textrm{一致性模型 } \; \mathcal{C} &\triangleq \set{\textrm{该模型所允许的所有系统执行}}}
%   \end{align*}
% 
%   \only<5->{\fignocaption{width = 0.20\textwidth}{figures/specs.png}}
% \end{frame}
%%%%%%%%%%%%%%%
\begin{frame}{维度二: 一致性实现机制}
  \begin{center}
	给定一致性模型 $\mathcal{C}$, 设计系统$\mathcal{S}$, 使得

	\pause
	\vspace{0.30cm}
	\textcolor{blue}{$\mathcal{S}$ 满足 $\mathcal{C}$}.

	即, 每个操作的返回值都合法.
  \end{center}

  % \[
  %   \forall e \in \mathcal{S}: e \in \mathcal{C}.
  % \]
  % \[
  %   \textrm{\it i.e., } \mathcal{S} \subseteq \mathcal{C}.
  % \]

  \pause
  \vspace{0.50cm}

  \begin{center}
	``多样化, 可调节''研究理念的挑战:
	\vspace{8pt}
	\begin{itemize}
	  \centering
	  \item 严格定义``多样化''
	  \item 系统实现``可调节''
	\end{itemize}
  \end{center}
\end{frame}
%%%%%%%%%%%%%%%
\begin{frame}{维度三: 一致性度量}
  \begin{center}
	给定一致性模型$\mathcal{C}$及系统$\mathcal{S}$,
  \end{center}

  \pause

  \begin{description}
	\centering
	\setlength{\itemsep}{5pt}
	\item[验证 (verify):] 执行 $e$ 是否满足 $\mathcal{C}$ 
	\item[量化 (quantify):] 协议 $\mathcal{P}$ 满足 $\mathcal{C}$ 的程度
  \end{description}
  % \fignocaption{width = 0.50\textwidth}{figures/qos-radio.png}

  \pause
  \vspace{0.50cm}

  \begin{center}
	``精细化, 可度量''研究理念的挑战:
	\vspace{8pt}
	\begin{description}
	  \centering
	  \item[验证:] 算法设计与分析
	  \item[量化:] 数学建模与求解
	\end{description}
  \end{center}
\end{frame}
%%%%%%%%%%%%%%%
