\section{研究方法}

%%%%%%%%%%%%%%%%%%%%%%%%%%%%%%
\subsection{理论模型:分布共享数据}

%%%%%%%%%%%%%%%
\begin{frame}{分布共享数据模型}
  \begin{columns}
	\column{0.50\textwidth}
	  \fignocaption{width = 0.60\textwidth}{figures/shared-data-clients.pdf}
	  \begin{center}
		共享数据系统 (single copy)
	  \end{center}
	\pause
	\column{0.50\textwidth}
	  \fignocaption{width = 0.70\textwidth}{figures/distributed-data-clients.pdf}
	  \begin{center}
		分布数据系统 (replicas)
	  \end{center}
  \end{columns}
\end{frame}
%%%%%%%%%%%%%%%
\begin{frame}{分布共享数据模型}
  \fignocaption{width = 0.48\textwidth}{figures/distributed-shared-data-clients.pdf}
  \begin{center}
	\textcolor{blue!80}{分布共享数据模型:} 在分布数据之上提供共享数据的假象
  \end{center}
\end{frame}
%%%%%%%%%%%%%%%
\begin{frame}{分布共享数据模型}
  \begin{center}
    \textcolor{cyan}{$p_0, p_1:$} 客户进程 \hspace{0.30cm} \textcolor{cyan}{$x, y:$} 共享变量 
  \end{center}
  \pause

  多进程并发读/写共享变量:
  
  \fignocaption{width = 0.55\textwidth}{figures/register-what-value.pdf}
  \pause

  \begin{center}
    \question{数据一致性问题: 读操作允许返回什么值?}
  \end{center}

  \answer{
  \[
    \text{不同一致性模型}
    \xrightleftharpoons[\text{定义}]{\,\text{规定}\,}
    \text{不同合法返回值}
  \]} 
\end{frame}
%%%%%%%%%%%%%%%
\begin{frame}{分布共享数据服务}
  \fignocaption{width = 0.40\textwidth}{figures/dsds.pdf}
\end{frame}
%%%%%%%%%%%%%%%
\begin{frame}{分布共享数据服务 (注)}
  \begin{columns}
	\column{0.60\textwidth}
	\begin{center}
	  分布共享内存模型 (多处理器系统)\\
	  \textcolor{blue}{[传统概念]}\\
	  \vspace{0.10cm} +\\ \vspace{0.10cm}
	  分布数据系统\\
	  \textcolor{blue}{[新平台]}
	  
	  \uncover<2->{
	  \color{red}\rule{\linewidth}{3pt}

	  \begin{center}
		\textcolor{blue}{新平台凸显应用价值观:}
		\begin{enumerate}
		  \centering
		  \item 精细化, 可度量
		  \item 多样化, 可调节
		\end{enumerate}
	  \end{center}
	  }
	\end{center}
	\column{0.40\textwidth}
	  \fignocaption{scale = 0.35}{figures/more-old-wine-new-bottle-poster.jpg}
  \end{columns}
\end{frame}
%%%%%%%%%%%%%%%
%%%%%%%%%%%%%%%%%%%%%%%%%%%%%%
\subsection{技术框架}

%%%%%%%%%%%%%%%
\begin{frame}{分布数据一致性问题}
  \begin{columns}
	\column{0.50\textwidth}
	  分布数据一致性问题:
	  \begin{itemize}
		\item [\textcolor{blue}{\bf \cmark}] 分布: partition + replication
		\item [\textcolor{red}{\bf \xmark}] 数据: 数据类型
		\item [\textcolor{red}{\bf \xmark}] 一致性: 关键问题
	  \end{itemize}
	\pause
	\column{0.50\textwidth}
	  \textcolor{blue}{数据类型:}
	  \begin{itemize}
		\item 单独的变量 (\textcolor{cyan}{$x,y$})
		\item 数据结构 (\textcolor{cyan}{\textsc{Set, List}})
		\item 事务 (\textcolor{cyan}{\textsc{Tx}})
	  \end{itemize}

	  \pause
	  \vspace{0.80cm}

	  \textcolor{blue}{一致性关键问题:}
	  \begin{itemize}
		\item 语义 (semantics; \textcolor{red}{\texttt{WHAT}})
		\item 机制 (mechanisms; \textcolor{red}{\texttt{HOW}})
		\item 度量 (measurements; \textcolor{red}{???})
	  \end{itemize}
  \end{columns}
\end{frame}
%%%%%%%%%%%%%%%
\begin{frame}{技术框架}
  % \fig{width = 0.75\textwidth}{figures/thesis-proposal-3d-framework-allinone.pdf}{数据一致性及保障技术研究框架}
  \todo{数据类型 + 一致性关键问题 技术框架}
\end{frame}
%%%%%%%%%%%%%%%
\begin{frame}{数据类型}
  \begin{columns}
	\column{0.60\textwidth}
	  数据类型: 从个体到群组
	  \begin{itemize}
		\setlength{\itemsep}{8pt}
		\item<1-> 单独读写变量 \textit{(read/write registers)}
		\item<2-> 事务对象 \textit{(transactions)}
		  \begin{itemize}
			\setlength{\itemsep}{4pt}
			\item 事务 $\triangleq$ 多个读写变量的操作序列
			\item 支持 ``all-or-none'' 写语义
			\item 易于开发并发应用
		  \end{itemize}
	  \end{itemize}
	\column{0.40\textwidth}
	  \uncover<1-> {
		\fignocaption{width = 0.50\textwidth}{figures/cassandra-logo.png}
	    \begin{center}
		  Key-Value Store
	    \end{center}
	  }
	  \uncover<2->{
		\fignocaption{width = 0.50\textwidth}{figures/figure-placeholder.png}
	  }
  \end{columns}
\end{frame}
%%%%%%%%%%%%%%%
\begin{frame}{一致性模型}
  一致性模型 \citeinbeamer{Steinke}{JACM}{04}:
  \begin{itemize}
	\item 规定读操作 (\textcolor{cyan}{\texttt{\it read}}) 所允许的返回值
  \end{itemize}

  \vspace{0.50cm}
  \fig{width = 0.50\textwidth}{figures/consistency-model-tree.png}{来自 \citeinbeamer{Bailis}{VLDB}{14}}
\end{frame}
%%%%%%%%%%%%%%%
\begin{frame}{一致性模型}
  一致性模型的\textcolor{blue}{集合}定义:
  \begin{align*}
	\uncover<2->{\textrm{系统执行 } \; e &\triangleq \textrm{该执行所产生的事件的序列} \\[3pt]}
	\uncover<3->{\textrm{分布式系统 } \; \mathcal{S} &\triangleq \set{\textrm{该系统的所有可能执行}} \\[3pt]}
	\uncover<4->{\textrm{一致性模型 } \; \mathcal{C} &\triangleq \set{\textrm{该模型所允许的所有系统执行}}}
  \end{align*}

  \uncover<2->{
  \begin{columns}
	\column{0.50\textwidth}
	  \fignocaption{width = 0.30\textwidth}{figures/three-gears.jpg}
	\column{0.50\textwidth}
	  \fignocaption{width = 0.30\textwidth}{figures/gps-marks.png}
  \end{columns}
  }
\end{frame}
%%%%%%%%%%%%%%%
\begin{frame}{一致性实现机制}
  \begin{columns}
	\column{0.60\textwidth}
	  \begin{center}
		给定一致性模型 $\mathcal{C}$, 设计维护算法及系统$\mathcal{S}$:
	  \end{center}

	  \[
		\forall e \in \mathcal{S}: e \in \mathcal{C}.
	  \]

	  \[
		\textrm{\it i.e., } \mathcal{S} \subseteq \mathcal{C}.
	  \]
	\column{0.40\textwidth}
	  \fignocaption{width = 0.60\textwidth}{figures/specs.png}
  \end{columns}

  \pause
  \vspace{0.30cm}

  \begin{columns}
	\column{0.60\textwidth}
	  ``多样化, 可调节''的挑战:
	  \begin{itemize}
		\item 兼容的混合一致性模型
		\item 手段之一: 参数化
	  \end{itemize}
	\column{0.40\textwidth}
	  \fignocaption{width = 0.45\textwidth}{figures/volumn-knob-cropped.png}
  \end{columns}
\end{frame}
%%%%%%%%%%%%%%%
\begin{frame}{一致性度量方法}
  \begin{columns}
	\column{0.60\textwidth}
	  给定分布式系统$\mathcal{S}$及某一致性模型$\mathcal{C}$, \\度量数据一致性:

	  \pause
	  \vspace{0.50cm}

	  对于$e \in \mathcal{S}$:
	  \begin{description}
		\setlength{\itemsep}{6pt}
		\item[验证 (verify):] $e \in \mathcal{C} \textcolor{red}{\mathbf{?}} \Rightarrow \set{0,1}$ 
		\item[量化 (quantify):] $e \in \mathcal{C} \textcolor{red}{\mathbf{?}} \Rightarrow (0,1)$ 
	  \end{description}
	\column{0.40\textwidth}
	  \fignocaption{width = 0.50\textwidth}{figures/qos-radio.png}
  \end{columns}

  \pause
  \vspace{0.80cm}

  \begin{columns}
	\column{0.50\textwidth}
	  ``精细化, 可度量''的挑战:
	  \begin{itemize}
		\item 问题复杂度分析, 算法设计
		\item 数学建模与分析
	  \end{itemize}
	\column{0.50\textwidth}
	  \fignocaption{width = 0.80\textwidth}{figures/george-box-model.jpg}
  \end{columns}
\end{frame}
%%%%%%%%%%%%%%%
