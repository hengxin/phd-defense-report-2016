\section{研究方法}

%%%%%%%%%%%%%%%%%%%%%%%%%%%%%%
\subsection{理论模型:分布共享数据}

%%%%%%%%%%%%%%%
\begin{frame}{数据的分布与共享}
  从共享到分布

  从分布到共享
\end{frame}
%%%%%%%%%%%%%%%
\begin{frame}{分布共享数据}
  \begin{center}
    \textcolor{cyan}{$x, y:$} 共享变量 \hspace{0.30cm} \textcolor{cyan}{$p_0, p_1:$} 
    客户进程
  \end{center}

  多进程并发提交 (读/写) 操作:
  
  \fignocaption{width = 0.55\textwidth}{figures/register-what-value.pdf}

  \begin{center}
    \question{问题: 读操作允许返回什么值?}
  \end{center}

  \answer{
  \[
    \text{不同一致性}
    \xrightleftharpoons[\text{定义}]{\,\text{规定}\,}
    \text{不同合法返回值}
  \]} 
\end{frame}
%%%%%%%%%%%%%%%
\begin{frame}{分布共享数据服务}
  \begin{columns}
	\column{0.50\textwidth}
	  \todo{图: 分布共享数据服务 (简化图)}
	  (数据)一致性模型是核心概念.
	\column{0.50\textwidth}
	  \begin{itemize}[<+->]
		\setlength\itemsep{10pt}
		\item 简化上层应用的开发
		\item 提供共享数据的抽象
		\item 屏蔽底层数据的分布性
	  \end{itemize}
  \end{columns}
\end{frame}
%%%%%%%%%%%%%%%
\begin{frame}{分布共享数据服务}
  \fig{width = 0.40\textwidth}{figures/dsm4replicas-model.pdf}{分布数据共享服务.}
\end{frame}
%%%%%%%%%%%%%%%
\begin{frame}{分布共享数据服务}
  \begin{center}
    \textcolor{blue}{\large 基本定位: 传统概念应用于新型平台}
  \end{center}

  \vspace{0.20cm}
  \begin{center}
    分布共享数据服务: 分布共享内存模型 + 分布数据系统
  \end{center}
\end{frame}
%%%%%%%%%%%%%%%
%%%%%%%%%%%%%%%%%%%%%%%%%%%%%%
\subsection{技术途径: 三维框架}

%%%%%%%%%%%%%%%
\begin{frame}{分布共享内存中的数据一致性问题}
      数据一致性问题的三个层面: 
      \begin{description}
	\item[1. 虚拟共享数据有什么?]
	  \begin{itemize}
	    \item 数据类型
	  \end{itemize}
	\item[2. 上层接口语义是什么?]
	  \begin{itemize}
	    \item 一致性模型
	  \end{itemize}
	\item[3. 底层消息传递为什么?]
	  \begin{itemize}
	    \item 一致性保障
	  \end{itemize}
      \end{description}
\end{frame}
%%%%%%%%%%%%%%%
\begin{frame}{研究框架}
  \fig{width = 0.75\textwidth}{figures/thesis-proposal-3d-framework-allinone.pdf}{数据一致性及保障技术研究框架}
\end{frame}
%%%%%%%%%%%%%%%
